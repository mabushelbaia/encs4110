% ---------- Base encoding & fonts ----------
\usepackage[T1]{fontenc}
\usepackage[utf8]{inputenc}
\usepackage{lmodern}
\usepackage{microtype}
\usepackage{accsupp}  % lets us control ActualText for copy/paste

\DeclareRobustCommand{\copyspace}{%
  \BeginAccSupp{ActualText= }%  % what the clipboard receives (a real U+0020)
  \kern0pt\ %                    % what is drawn (a normal space)
  \EndAccSupp
}
% ---------- Graphics & links ----------
\usepackage{graphicx}
\usepackage[dvipsnames]{xcolor}
\usepackage{hyperref}
\hypersetup{
  colorlinks=true,
  linkcolor=blue!60!black,
  urlcolor=blue!60!black,
  citecolor=blue!60!black
}

% ---------- Lists, tables, floats ----------
\usepackage{enumitem}                % for [nosep]
\usepackage{booktabs}                % \toprule etc.
\usepackage{float}                   % [H] placement
\usepackage{tabularx,array}
\renewcommand{\arraystretch}{1.15}

% ---------- KOMA (safe if class is scrreprt/scrartcl) ----------
\makeatletter
\@ifundefined{KOMAClassName}{}{%
  \KOMAoptions{
    DIV=14,
    BCOR=0mm,
    headinclude=true,
    footinclude=false
  }%
}
\makeatother

% ---------- Numbering & ToC ----------
\setcounter{secnumdepth}{3}
\setcounter{tocdepth}{3}             % show down to subsubsection in ToC
\renewcommand\thesection{\arabic{section}}
\renewcommand\thesubsection{\thesection.\arabic{subsection}}
\renewcommand\thesubsubsection{\thesubsection.\arabic{subsubsection}}

% ---------- Small helpers ----------
\newcommand{\keil}{\textbf{Keil uVision5}}
\providecommand{\tightlist}{\setlength{\itemsep}{0pt}\setlength{\parskip}{0pt}}

% ---------- Listings: ARM assembly (colors + top/bottom rules only) ----------
\usepackage{listings}
\usepackage{listingsutf8}

% Palette
\definecolor{AsmMnemonic}{HTML}{005CC5}   % blue
\definecolor{AsmDirective}{HTML}{6F42C1}  % purple
\definecolor{AsmCond}{HTML}{D73A49}       % red
\definecolor{AsmRegister}{HTML}{22863A}   % green
\definecolor{AsmComment}{HTML}{6A737D}    % gray
\definecolor{AsmLineNo}{HTML}{9AA0A6}     % light gray
\definecolor{AsmRule}{HTML}{C0C4C8}       % rule color

% Language
\lstdefinelanguage{Assembler}{
  % Mnemonics (include S-suffixed forms and conditional branch forms)
  morekeywords={
    ADC,ADCS,ADD,ADDS,ADR,AND,ANDS,ASR,ASRS,
    B,BEQ,BNE,BCS,BHS,BCC,BLO,BMI,BPL,BVS,BVC,BHI,BLS,BGE,BLT,BGT,BLE,BAL,
    BL,BLX,BX,
    CBNZ,CBZ,CLZ,CMP,CMN,
    EOR,EORS,
    ISB,
    LDM,LDR,LDRB,LDRH,
    LSL,LSLS,LSR,LSRS,
    MOV,MOVS,MOVT,MOVW,MSR,MRS,
    MUL,MULS,
    NOP,
    ORR,ORRS,
    POP,PUSH,
    REV,REV16,REVSH,
    ROR,RORS,
    SBC,SBCS,
    SEV,
    STM,STR,STRB,STRH,
    SUB,SUBS,
    TBB,TBH,TST,
    UXTB,UXTH,UXTB16,
    WFE,WFI,
    YIELD,
    BKPT
  },
  % Directives
  morekeywords=[2]{AREA,ALIGN,DCD,DCB,DCW,EXPORT,IMPORT,ENTRY,END,SPACE,KEEP,WEAK,LTORG,PROC,ENDP},
  % Condition-code tokens (when written separately; kept for completeness)
  morekeywords=[3]{EQ,NE,CS,HS,CC,LO,MI,PL,VS,VC,HI,LS,GE,LT,GT,LE,AL},
  morecomment=[l]{;},
  sensitive=true
}
\lstdefinelanguage[ARM]{Assembler}[]{Assembler}{}


\lstdefinestyle{labcode}{
  language={[ARM]Assembler},
  inputencoding=utf8,
  basicstyle=\ttfamily\small,  % inconsolata will be used
  columns=fixed,               % fixed-width columns -> real spaces in PDF
  keepspaces=true,             % preserve all spaces (incl. leading)
  tabsize=4,                   % set to the width you expect; or convert tabs to spaces
  showstringspaces=false,
  upquote=true,
  keywordstyle={\color{AsmMnemonic}\bfseries},
  keywordstyle=[2]{\color{AsmDirective}\bfseries},
  keywordstyle=[3]{\color{AsmCond}\bfseries},
  commentstyle={\itshape\color{AsmComment}},
  emph={R0,R1,R2,R3,R4,R5,R6,R7,R8,R9,R10,R11,R12,SP,LR,PC,
        APSR,IP,PSP,MSP,xPSR},
  emphstyle={\color{AsmRegister}\bfseries},
  showstringspaces=false,
  tabsize=2,
  keepspaces=true,
  columns=fullflexible,
  % line breaking
  breaklines=true,
  breakatwhitespace=true,
  % frame: top & bottom only
  frame=lines,
  rulecolor={\color{AsmRule}},
  xleftmargin=2em,
  framexleftmargin=2em,
  % line numbers
  % line numbers disabled
  % float control
  float,
  floatplacement=H,
  captionpos=b,
}
\lstset{style=labcode}
\usepackage{amsmath}
\usepackage{tcolorbox}
\usepackage{tikz}
\usetikzlibrary{positioning,arrows.meta,decorations.pathreplacing}
\usepackage{subcaption} % or subfig if you prefer
% ---------- Chapter-level ToCs ----------

% ---------- Tables of contents ----------
\usepackage{etoc}  % local ToCs that inherit the same style as \tableofcontents
\usepackage{titletoc}  % for \titlecontents command
% ---------- Cover page config ----------
% If you use an SVG logo, keep this. If you prefer PNG/PDF, you may omit.
\usepackage{svg} % requires --shell-escape or inkscape; else convert logo to PDF/PNG

% Editable cover fields

% ---- Technical manual title page helpers ----
% Logo is expected at assets/<name>.svg; adjust if needed.
\svgpath{{resources/}} % where the logo lives

\providecommand{\LOGONAME}{birzeit-logo} % base name, no extension
\providecommand{\DEPARTMENT}{Department of Electrical \& Computer Engineering}
\providecommand{\COURSENUM}{ENCS4110}
\providecommand{\COURSENAME}{Computer Design Laboratory}
% ---- Cover defaults (override as needed) ----
\providecommand{\ManualSubtitle}{Laboratory Manual}

% Semantic version pieces (or set \ManualVersion directly)
\providecommand{\ManualVersionMajor}{1}
\providecommand{\ManualVersionMinor}{0}
\providecommand{\ManualVersionPatch}{0}
\providecommand{\ManualVersion}{v\ManualVersionMajor.\ManualVersionMinor.\ManualVersionPatch}
\newcommand{\MonthYear}{%
  \ifcase\month
  \or January\or February\or March\or April\or May\or June%
  \or July\or August\or September\or October\or November\or December
  \fi~\the\year
}
% Local ToC etc. can stay as-is. Ensure svg path is set:
\makeatletter
\newcommand{\maketechnicaltitlepage}{%
  \begin{titlepage}
    \thispagestyle{empty}

    % --- Top: logo + department + course ---
    \vspace*{1.2cm}
    \begin{center}
      \IfFileExists{assets/\LOGONAME.pdf}
        {\includegraphics[width=.80\textwidth]{assets/\LOGONAME.pdf}}
        {\includesvg[width=.80\textwidth,inkscapelatex=false]{\LOGONAME}}

      \vspace{0.3cm}
      {\large \DEPARTMENT\par}
      {\large \COURSENUM\;--\;\COURSENAME\par}
    \end{center}

    % --- Middle: centered Title + Subtitle ONLY ---
    \vspace{1.5cm}
    \begin{center}
      {\Huge\bfseries \@title\par}
      \vspace{0.35cm}
      {\Large \ManualSubtitle\par}
    \end{center}
    \vfill

    % --- Bottom-left: Version + Date ---
    \begin{flushleft}
      {\large \textbf{Version:} \ManualVersion}\\[0.2cm]
        {\large \textbf{Date:} \MonthYear}
    \end{flushleft}
  \end{titlepage}%
}
\makeatother

% ---- titletoc formatting: Tufte-ish, robust ----
\usepackage{titletoc}

\titlecontents{part}%
  [0pt]% distance from left margin
  {\addvspace{0.25\baselineskip}}% above (global formatting of entry)
  {\allcaps{Part~\thecontentslabel}\quad}% before w/ label (label = ``Part I'')
  {\allcaps{Part~\thecontentslabel}\allcaps}% before w/o label
  {}% filler and page (leaders and page num)
  [\vspace*{0.5\baselineskip}]% after

\titlecontents{chapter}%
    [4em]% distance from left margin
    {}% above (global formatting of entry)
    {\contentslabel{2em}\textit}% before w/ label (label = ``Chapter 1'')
    {\hspace{0em}\textit}% before w/o label
    {\qquad\thecontentspage}% filler and page (leaders and page num)
    [\vspace*{0.5\baselineskip}]% after
%%%% End additional code by Kevin Godby


% Silent part: adds a numbered Part entry to the ToC, prints nothing
% --- ToC-only “Part” entry (no heading printed, no hyperlink in ToC) ---
\makeatletter
\newcommand{\toconlypart}[1]{%
  \clearpage                 % keep if you want a clean break; remove if not
  \refstepcounter{part}%     % step the Part counter (I, II, …)
  % Write a plain (non-hyperlinked) contents line: 4th arg = link target -> empty
  \addtocontents{toc}{%
    \protect\contentsline{part}{\protect\numberline{\thepart}#1}{\thepage}{}%
  }%
  % Optional: set header marks if you use \leftmark/\rightmark
  \markboth{Part \thepart\quad #1}{}%
}
\makeatother


\usepackage{bytefield}
\usepackage[table]{xcolor} % for light gray shading
\usepackage{needspace}


\usepackage{graphicx} % for \resizebox
\usepackage{calc}     % for \widthof
\usepackage{bytefield}

% measure once
\newlength{\pfbit}
\settowidth{\pfbit}{\tiny PF0} % or {\scriptsize PF0} if you prefer
