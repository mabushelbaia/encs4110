\chapter{TM4C123 Microcontroller: Architecture and Digital Outputs}

\section*{Learning Objectives}
After completing this experiment, you will be able to:
\begin{itemize}[nosep]
  \item Identify the main components and subsystems of the TM4C123 microcontroller and understand their roles within an embedded system.
  \item Recognize the organization of the ARM Cortex-M4 core, memory map, and peripheral buses.
  \item Understand the role of core peripherals including NVIC, SysTick, and System Control Block (SCB).
  \item Configure and control digital output pins using the General-Purpose Input/Output (GPIO) module.
  \item Write Assembly programs to drive on-board LEDs through direct register access.
  \item Relate register operations to physical hardware behavior and verify functionality through observation on the LaunchPad board.
\end{itemize}

\section*{Experiment Overview}
This experiment introduces the architecture of the TM4C123 microcontroller and its most fundamental feature — the ability to interface with the outside world through digital I/O pins.

\noindent You will first explore the internal structure of the device, including the ARM Cortex-M4 core, core peripherals (NVIC, SysTick, SCB), buses, memory regions, and peripheral interconnect. Then, you will apply this understanding by writing Assembly programs that toggle the on-board LEDs connected to \texttt{PORTF}.

\noindent In this experiment, you will:
\begin{itemize}[nosep]
  \item Understand how peripheral registers are mapped into memory and accessed by the CPU.
  \item Learn about Cortex-M4 core peripherals and their functions.
  \item Learn the initialization sequence required to enable and configure GPIO ports.
  \item Develop Assembly programs that control LEDs using software delays.
  \item Form the foundation for subsequent experiments involving interrupts, timers, and analog peripherals.
\end{itemize}

By the end of this lab, you will understand the TM4C123 architecture, Cortex-M4 core peripherals, be able to configure GPIO ports for digital output, and control external hardware through register manipulation in Assembly.

\newpage
\etocsetnexttocdepth{subsubsection}
\localtableofcontents
\bigskip
\newpage

\section{Theoretical Background}

\subsection{TM4C123 Microcontroller Architecture}

The TM4C123GH6PM microcontroller is built around the ARM Cortex-M4F processor core, implementing the ARMv7-M architecture. It integrates the CPU core, memory, peripherals, and I/O interfaces on a single chip — a complete system-on-chip (SoC) solution for embedded applications.

\subsubsection{Core Components}

The TM4C123 contains the following major components:
\begin{itemize}[nosep]
  \item \textbf{ARM Cortex-M4F Processor Core}: 32-bit RISC processor with hardware Floating-Point Unit (FPU)
  \item \textbf{Memory}: 256 KB Flash, 32 KB SRAM, 2 KB EEPROM
  \item \textbf{System Buses}: Advanced High-Performance Bus (AHB) and Advanced Peripheral Bus (APB)
  \item \textbf{Core Peripherals}: NVIC, SysTick Timer, Memory Protection Unit (MPU), System Control Block (SCB)
  \item \textbf{General-Purpose Timers}: Six 16/32-bit and six 32/64-bit timers
  \item \textbf{Communication Interfaces}: UART, I\textsuperscript{2}C, SSI (SPI), CAN
  \item \textbf{Analog Peripherals}: 12-bit ADC with 12 channels, analog comparators
  \item \textbf{GPIO Ports}: Six ports (A-F) with up to 43 programmable pins
\end{itemize}

\subsection{ARM Cortex-M4 Core Peripherals}

The ARM Cortex-M4 processor includes several \textbf{core peripherals} that are common across all Cortex-M4-based microcontrollers.  
They are tightly integrated with the CPU and provide essential system functions such as interrupt handling and timing.

\subsubsection{Nested Vectored Interrupt Controller (NVIC)}

The NVIC manages all interrupt and exception handling.  
It supports up to 240 interrupt sources (138 on the TM4C123), hardware priority levels, and automatic context saving for efficient servicing.  
Key features include nesting, tail-chaining, and late-arrival handling for minimal interrupt latency.

\noindent Important NVIC registers (base address \texttt{0xE000E100}):
\begin{itemize}[nosep]
  \item \texttt{NVIC\_ENx} — Set-Enable  
  \item \texttt{NVIC\_DISx} — Clear-Enable  
  \item \texttt{NVIC\_PRIx} — Priority Configuration  
  \item \texttt{NVIC\_ACTIVEx} — Active Status
\end{itemize}

\subsubsection{SysTick Timer}

The SysTick timer is a simple 24-bit down-counter integrated in the core.  
It provides a consistent time base for system delays or periodic tasks.  
Many operating systems use it for time-keeping or scheduling.

\noindent SysTick registers (base address \texttt{0xE000E010}):
\begin{itemize}[nosep]
  \item \texttt{STCTRL} — Control and Status  
  \item \texttt{STRELOAD} — Reload Value  
  \item \texttt{STCURRENT} — Current Counter Value
\end{itemize}

\noindent Other core peripherals such as the System Control Block (SCB) are present but not covered in this course.
\bigskip

\subsection{Memory Map}

The Cortex-M4 processor uses a unified 4 GB address space for all code, data, and peripherals.  
Every peripheral and memory region occupies a unique address range, allowing direct access through normal load and store instructions.

\begin{table}[H]
\centering
\small
\begin{tabular}{lll}
\toprule
\textbf{Region} & \textbf{Address Range} & \textbf{Description} \\
\midrule
Flash Memory & 0x00000000 - 0x0003FFFF & Program storage (256 KB) \\
SRAM & 0x20000000 - 0x20007FFF & On-chip data memory (32 KB) \\
Peripherals & 0x40000000 - 0x400FFFFF & Peripheral registers \\
GPIO Ports & 0x40004000 - 0x40025FFF & GPIO A-F registers \\
Core Peripherals & 0xE0000000 - 0xE00FFFFF & NVIC, SysTick, SCB, etc. \\
\bottomrule
\end{tabular}
\caption{Simplified TM4C123 Memory Map}
\end{table}

\noindent
Each peripheral's registers are \textbf{memory-mapped}, meaning they are accessed just like variables in memory.  
For example, writing to address \texttt{0x400253FC} directly updates the GPIO Port F data register.


\subsection{General-Purpose Input/Output (GPIO)}

GPIO (General-Purpose Input/Output) ports form the primary interface between the microcontroller and external devices such as LEDs, switches, and sensors. Each pin can be configured as either an input or an output.

When a pin is an \textbf{output}, software drives it high or low to control external hardware (e.g., turn an LED on/off).  
When a pin is an \textbf{input}, software reads its logic level (e.g., a button press).

\subsubsection{GPIO Ports Overview}

The TM4C123GH6PM provides six GPIO ports (A-F), each with up to eight programmable pins. Not all pins are available on the LaunchPad, and some are reserved for debugging or special functions.

\begin{table}[H]
\centering
\small
\renewcommand{\arraystretch}{1.2}
\begin{tabular}{cll}
\toprule
\textbf{Port} & \textbf{Pins Available} & \textbf{Notes} \\
\midrule
Port A & PA0-PA7 & UART0 (PA0, PA1) shared with USB debug interface \\
Port B & PB0-PB7 & I\textsuperscript{2}C0, SSI2, ADC channels \\
Port C & PC0-PC7 & PC0-PC3 used for JTAG (avoid modification) \\
Port D & PD0-PD7 & PD7 requires unlock for GPIO use \\
Port E & PE0-PE5 & ADC and UART5 functionality available \\
Port F & PF0-PF4 & On-board LEDs (PF1-PF3) and switches (PF0, PF4) \\
\bottomrule
\end{tabular}
\caption{GPIO Port Summary — TM4C123GH6PM}
\end{table}

\noindent
Each port exposes its own control and data registers, allowing independent configuration and operation.
\bigskip

\subsubsection{Memory-Mapped GPIO Registers}

GPIO modules are accessed via \textbf{memory-mapped registers}: fixed addresses that the CPU reads/writes with standard \texttt{LDR}/\texttt{STR} instructions.

\noindent
Every port has a \textbf{base address}; key registers sit at fixed offsets from that base.

\begin{table}[H]
\centering
\small
\begin{tabular}{cl}
\toprule
\textbf{Port} & \textbf{Base Address} \\
\midrule
Port A & 0x40004000 \\
Port B & 0x40005000 \\
Port C & 0x40006000 \\
Port D & 0x40007000 \\
Port E & 0x40024000 \\
Port F & 0x40025000 \\
\bottomrule
\end{tabular}
\caption{GPIO Port Base Addresses (TM4C123GH6PM)}
\end{table}

\noindent
In this experiment we use Port~F, so:
\[
\texttt{GPIODIR} = \texttt{0x40025400},\quad
\texttt{GPIODEN} = \texttt{0x4002551C},\quad
\texttt{GPIODATA} = \texttt{0x400253FC}.
\]
\bigskip

\subsubsection{Address Masking in \texttt{GPIODATA}}

The \texttt{GPIODATA} register supports \textbf{address masking}: address bits [9:2] form a bit mask that selects which pins are affected.

\begin{itemize}[nosep]
  \item \textbf{Full access:} Base + \texttt{0x3FC} — all 8 pins.
  \item \textbf{Masked access:} Base + (\texttt{mask} $\ll$ 2) — only the pins in \texttt{mask}.
\end{itemize}

\noindent
Example (PF1 only):
\[
\texttt{0x40025000 + (0x02 << 2) = 0x40025008}.
\]
This is \textbf{address masking} (often confused with “bit-banding”); it enables fast, selective reads/writes to individual pins.
\bigskip

\subsubsection{Protected and Special-Function Pins}

Some pins are locked or reserved at reset due to critical roles:

\begin{itemize}[nosep]
  \item \textbf{PF0} doubles as the Non-Maskable Interrupt (NMI) input and is locked. To use it as GPIO, write \texttt{0x4C4F434B} to \texttt{GPIO\_LOCK} and set the desired bits in \texttt{GPIO\_CR}.
  \item \textbf{PC0-PC3} carry the JTAG debug interface and should not be reassigned in normal operation.
\end{itemize}

\noindent
Always confirm pin availability in the datasheet before repurposing special-function pins.
\bigskip

\subsubsection{Port F on the LaunchPad}

Port~F connects to the on-board RGB LEDs and user push buttons:

\begin{table}[H]
\centering
\small
\renewcommand{\arraystretch}{1.2}
\begin{tabular}{clll}
\toprule
\textbf{Pin} & \textbf{Function} & \textbf{Connected Component} & \textbf{Active Level} \\
\midrule
PF1 & Output & Red LED & Active High \\
PF2 & Output & Blue LED & Active High \\
PF3 & Output & Green LED & Active High \\
PF0 & Input (locked) & SW2 (Push Button) & Active Low (enable pull-up) \\
PF4 & Input & SW1 (Push Button) & Active Low (enable pull-up) \\
\bottomrule
\end{tabular}
\caption{GPIO Port F Pin Assignments — TM4C123 LaunchPad}
\end{table}

\noindent
LEDs are \textbf{active-high} (write 1 to turn on). Push buttons pull to ground when pressed, so inputs read \textbf{0 when pressed} and require internal pull-ups.
\bigskip

\subsubsection{GPIO Configuration Workflow}

In practice, GPIO usage has two phases:

\begin{itemize}[nosep]
  \item \textbf{Initialization (setup, runs once):} enable the port clock, unlock protected pins (if needed), set direction, and enable digital function.
  \item \textbf{Runtime (operation, repeats):} read inputs or write outputs by accessing \texttt{GPIODATA} in the main loop or in interrupt service routines (ISRs).
\end{itemize}

\noindent
A typical structure is:

\begin{lstlisting}[language=C,caption={Typical GPIO program structure (setup vs. runtime)}]
InitGPIO();                // RCGCGPIO, optional LOCK/CR, DIR, DEN
while (1) {                // Runtime phase
    // Read buttons (inputs) and/or drive LEDs (outputs)
    // Optionally, some I/O is handled inside ISRs
}
\end{lstlisting}

\medskip
\noindent
The detailed register-level steps (clock enable, unlock, direction, digital enable, and data access) are demonstrated in the following subsections.

\newpage
\subsubsection*{Step 1 — Enable the GPIO Port Clock}

\noindent\textbf{Register:} \texttt{RCGCGPIO} — Run Mode Clock Gating Control for GPIO (\texttt{0x400FE608})

\noindent
Before accessing any GPIO port, its clock must be enabled.  
The \texttt{RCGCGPIO} register controls the clock to all GPIO modules.  
Each bit corresponds to a single port (A-F). Writing a '1' to a bit enables the clock to that port, while '0' disables it.

\begin{figure}[H]
\centering
\begin{bytefield}[bitwidth=1.1em,bitheight=2.7ex,
                  boxformatting={\centering\small},
                  endianness=big, bitwidth=\widthof{\tiny~PF~}]{32}
\bitheader{31,5,4,3,2,1,0}\\
\bitbox{26}[bgcolor=gray!10]{\textit{Reserved}} &
\bitbox{1}{\tiny{PF}} &
\bitbox{1}{\tiny{PE}} &
\bitbox{1}{\tiny{PD}} &
\bitbox{1}{\tiny{PC}} &
\bitbox{1}{\tiny{PB}} &
\bitbox{1}{\tiny{PA}}
\end{bytefield}
\caption{RCGCGPIO Register (\texttt{0x400FE608}) — GPIO Clock Control}
\end{figure}

\noindent
To activate Port~F, bit~5 must be set to~1 as shown below.

\begin{lstlisting}[caption={Enable clock for Port F}]
        LDR     R1, =0x400FE608     ; RCGCGPIO register
        LDR     R0, [R1]
        ORR     R0, R0, #0x20       ; Set bit 5 (Port F)
        STR     R0, [R1]
\end{lstlisting}
\noindent
Once the clock is enabled, a short delay or status polling ensures the peripheral is ready before further configuration.
\bigskip


\subsubsection*{Step 2 — Unlock Protected Pins}
\noindent\textbf{Registers:} \texttt{GPIO\_LOCK} (\texttt{0x40025520}), \texttt{GPIO\_CR} (\texttt{0x40025524})

\noindent
Certain pins such as PF0 are protected because they share critical alternate functions (e.g., the NMI input).  
To modify these pins, the port must be unlocked by writing the key value \texttt{0x4C4F434B} (“LOCK”) into the \texttt{GPIO\_LOCK} register.  
The \texttt{GPIO\_CR} register (Commit Register) then determines which pins can be altered.

\begin{figure}[H]
\centering
\begin{bytefield}[bitheight=2.7ex,
  boxformatting={\centering\small},endianness=big, bitwidth=\widthof{\tiny~PF~}]{32}
\bitheader{31,8,7,0}\\
\bitbox{24}[bgcolor=gray!10]{\textit{Reserved}} &
\bitbox{8}{\tiny Commit Mask}
\end{bytefield}
\caption{GPIO\_CR Register — Commit Control}
\end{figure}

\noindent
The following code unlocks Port~F and enables modification of all pins.

\begin{lstlisting}[caption={Unlock Port F}]
        LDR     R1, =0x40025520     ; GPIO_LOCK
        LDR     R0, =0x4C4F434B     ; Unlock key
        STR     R0, [R1]            
        LDR     R1, =0x40025524     ; GPIO_CR
        MOV     R0, #0xFF           ; Allow changes to all pins
        STR     R0, [R1]
\end{lstlisting}
\noindent
In this experiment, PF0 is not used, so unlocking is optional. After unlocking, configuration registers such as direction and digital enable can be safely modified.
\bigskip


\subsubsection*{Step 3 — Configure Pin Direction}
\noindent\textbf{Register:} \texttt{GPIODIR} (\texttt{0x40025400}) — GPIO Direction Control (Port~F)

\noindent
The \texttt{GPIODIR} register determines whether each GPIO pin functions as an input or an output.  
Writing '0' configures the pin as an input, while '1' configures it as an output.

\begin{figure}[H]
\centering
\begin{bytefield}[bitwidth=\widthof{\tiny~PF~},bitheight=2.7ex,
  boxformatting={\centering\small},endianness=big]{32}
\bitheader{31,8,7,6,5,4,3,2,1,0}\\
\bitbox{24}[bgcolor=gray!10]{\textit{Reserved}} &
\bitbox{1}{\tiny F7} &
\bitbox{1}{\tiny F6} &
\bitbox{1}{\tiny F5} &
\bitbox{1}{\tiny F4} &
\bitbox{1}{\tiny F3} &
\bitbox{1}{\tiny F2} &
\bitbox{1}{\tiny F1} &
\bitbox{1}{\tiny F0}
\end{bytefield}
\caption{GPIODIR Register (Port F) — Direction Control}
\end{figure}

\noindent
For the on-board RGB LED, PF1-PF3 must be configured as outputs.  
Bits 1-3 are therefore set to '1'.

\begin{lstlisting}[caption={Set PF1, PF2, PF3 as outputs}]
        LDR     R1, =0x40025400     ; GPIODIR register (Port F)
        LDR     R0, [R1]
        ORR     R0, R0, #0x0E       ; Set PF1, PF2, PF3 as outputs
        STR     R0, [R1]
\end{lstlisting}
\noindent
Unmodified bits remain unchanged, allowing input pins (such as PF0 or PF4) to retain their default configuration.
\bigskip


\subsubsection*{Step 4 — Enable Digital Functionality}
\noindent\textbf{Register:} \texttt{GPIODEN} (\texttt{0x4002551C}) — Digital Enable Register (Port~F)

\noindent
Each GPIO pin can serve analog or digital functions.  
The \texttt{GPIODEN} register enables the digital circuitry for selected pins.  
Pins configured as digital inputs or outputs must have their corresponding bits set to '1'.

\begin{figure}[H]
\centering
\begin{bytefield}[bitwidth=\widthof{\tiny~PF~},bitheight=2.7ex,
  boxformatting={\centering\small},endianness=big]{32}
\bitheader{31,8,7,6,5,4,3,2,1,0}\\
\bitbox{24}[bgcolor=gray!10]{\textit{Reserved}} &
\bitbox{1}{\tiny F7} &
\bitbox{1}{\tiny F6} &
\bitbox{1}{\tiny F5} &
\bitbox{1}{\tiny F4} &
\bitbox{1}{\tiny F3} &
\bitbox{1}{\tiny F2} &
\bitbox{1}{\tiny F1} &
\bitbox{1}{\tiny F0}
\end{bytefield}
\caption{GPIODEN Register (Port F) — Digital Enable}
\end{figure}

\noindent
Bits~1-3 are set to enable PF1-PF3 as digital outputs for the LED.

\begin{lstlisting}[caption={Enable digital function for PF1-PF3}]
        LDR     R1, =0x4002551C     ; GPIODEN register (Port F)
        LDR     R0, [R1]
        ORR     R0, R0, #0x0E       ; Enable PF1-PF3 as digital pins
        STR     R0, [R1]
\end{lstlisting}
\noindent
Failing to enable \texttt{GPIODEN} leaves the pins electrically inactive even if their direction is set.
\bigskip


\subsubsection*{Step 5 — Write to the Data Register}
\noindent\textbf{Register:} \texttt{GPIODATA} (\texttt{0x400253FC}) — Data Input/Output Register (Port~F)

\noindent
The \texttt{GPIODATA} register reflects the current logic levels on all GPIO pins.  
Writing to a bit drives the corresponding output high ('1') or low ('0').  
Bits 1-3 correspond to the RGB LED pins on the LaunchPad:  
PF1 = Red, PF2 = Blue, PF3 = Green.

\begin{figure}[H]
\centering
\begin{bytefield}[bitwidth=\widthof{\tiny ~PF~},bitheight=2.7ex,
  boxformatting={\centering\small},endianness=big]{32}
\bitheader{31,8,7,6,5,4,3,2,1,0}\\
\bitbox{24}[bgcolor=gray!10]{\textit{Reserved}} &
\bitbox{1}{\tiny F7} &
\bitbox{1}{\tiny F6} &
\bitbox{1}{\tiny F5} &
\bitbox{1}{\tiny F4} &
\bitbox{1}{\tiny F3} &
\bitbox{1}{\tiny F2} &
\bitbox{1}{\tiny F1} &
\bitbox{1}{\tiny F0}
\end{bytefield}
\caption{GPIODATA Register (Port F) — Data Output/Input}
\end{figure}

\noindent
The example below drives PF1 (Red) and PF3 (Green) simultaneously to produce a yellow color.

\begin{lstlisting}[caption={Turn on Yellow LED (Red + Green = PF1 + PF3)}]
    LDR     R1, =0x400253FC     ; GPIODATA register (Port F)
    MOV     R0, #0x0A           ; Set PF1 (Red) and PF3 (Green) = Yellow
    STR     R0, [R1]
\end{lstlisting}
\noindent
By writing different bit combinations, various LED colors can be generated:
\newcommand{\ledon}[1]{\tikz\fill[#1,draw=black] (0,0) circle (0.6ex);}
\newcommand{\ledoff}{\tikz\draw[gray!70] (0,0) circle (0.6ex);}

\begin{table}[H]
\centering
\small

\renewcommand{\arraystretch}{1.2}
\begin{tabular}{lccccl}
\toprule
\textbf{Color} & \textbf{Red (PF1)} & \textbf{Blue (PF2)} & \textbf{Green (PF3)} & \textbf{Hex} & \textbf{Combination} \\
\midrule
Red      & \ledon{red}    & \ledoff & \ledoff & 0x02 & Red only \\
Blue     & \ledoff        & \ledon{blue} & \ledoff & 0x04 & Blue only \\
Green    & \ledoff        & \ledoff & \ledon{green!70!black} & 0x08 & Green only \\
Yellow   & \ledon{red}    & \ledoff & \ledon{green!70!black} & 0x0A & Red + Green \\
Cyan     & \ledoff        & \ledon{blue} & \ledon{green!70!black} & 0x0C & Blue + Green \\
Magenta  & \ledon{red}    & \ledon{blue} & \ledoff & 0x06 & Red + Blue \\
White    & \ledon{red}    & \ledon{blue} & \ledon{green!70!black} & 0x0E & All on \\
\bottomrule
\end{tabular}
\caption{LED Color Combinations on TM4C123 LaunchPad (PF1-PF3)}
\label{tab:led-colors}
\end{table}
\newpage
\section{Procedure}
\subsection{Examples}
The following examples illustrate the complete process of initializing Port~F and controlling the on-board LEDs using Assembly language.
\subsubsection{Example 1 — Simple LED Blink}
This example initializes Port~F and toggles the Red LED (PF1) on and off using address masking with a software delay.
Address Masking for PF1:
\[
\texttt{0x40025000 + (0x02 << 2) = 0x40025008}
\]
\Needspace{50\baselineskip} % reserve space for the block; else start a new page
\lstinputlisting[caption={Assembly code to blink the Red LED (PF1) on the TM4C123 LaunchPad}]{snippets/gpio/example1.asm}

\subsubsection{Example 2 — Cycle Through RGB Colors}
This example extends the previous one by cycling through RGB LED colors (Red, Green, Blue) using a software delay.
Address Masking for PF1, PF2, PF3:
\[
\texttt{0x40025000 + ((0x02 | 0x04 | 0x08) << 2) = 0x40025038}
\]
\Needspace{55\baselineskip} % reserve space for the block; else start a new page
\lstinputlisting[caption={Assembly code to cycle through RGB colors on the TM4C123 LaunchPad}]{snippets/gpio/example2.asm}
\subsection{Tasks}

\subsubsection{Task 1 — Adjust the Blink Rate}
Modify the delay routine in \textbf{Example 1} to change the blinking speed of the Red LED.  
Experiment with different delay values until the LED blinks at approximately \textbf{1 Hz} (about one second ON and one second OFF).

\medskip

\subsubsection{Task 2 — Cycle Through Multiple Colors}
Expand \textbf{Example 2} to include additional colors by combining the Red, Green, and Blue LEDs.  
Create a program that automatically cycles through all color combinations listed in Table~\ref{tab:led-colors}. \\

\smallskip

\noindent\textit{Hint:} Use a loop to step through the color sequence repeatedly instead of writing separate code for each color.
