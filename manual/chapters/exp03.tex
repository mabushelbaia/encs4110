\chapter{ARM Cortex-M4 Flow Control, Procedures, and Stack}

\section*{Learning Objectives}
\begin{itemize}[nosep]
  \item Implement conditional and unconditional branching using ARM branch instructions.
  \item Design and implement loops (for, while) using compare and branch instructions.
  \item Create and call procedures with proper parameter passing and return mechanisms.
  \item Manage the stack for local variables, parameter passing, and nested procedure calls.
  \item Apply the ARM Procedure Call Standard (AAPCS) for register usage and calling conventions.
\end{itemize}

\section*{Experiment Overview}
This experiment explores program control flow, procedure implementation, and stack management in ARM Cortex-M4 assembly. You will learn to control program execution using branches and loops, create reusable code modules through procedures, and manage memory efficiently using the stack.

\noindent You will:
\begin{itemize}[nosep]
  \item Implement various loop constructs and conditional statements in assembly.
  \item Write procedures that follow standard calling conventions.
  \item Handle nested procedure calls and parameter passing.
  \item Manage stack operations for local variables and return addresses.
\end{itemize}
After completing this experiment, you will understand how to use branches and loops in ARM assembly, write simple procedures that follow standard conventions, and manage the stack for function calls.

\newpage
\tableofcontents
\newpage
\section{Theoretical Background}

\subsection{Flow Control Instructions}
Flow control instructions alter the sequential execution of instructions by changing the program counter (PC). These instructions enable the implementation of conditional statements, loops, and procedure calls that are fundamental to structured programming.

\subsubsection{Branch Instructions}
Branch instructions are the primary mechanism for implementing flow control in ARM assembly. They modify the program counter to jump to different parts of the code based on conditions or unconditionally.

\begin{table}[H]
\centering
\caption{ARM Cortex-M4 Branch Instructions}
\small
\begin{tabularx}{\linewidth}{@{}l l X@{}}
\toprule
\textbf{Instr.} & \textbf{Syntax} & \textbf{Description / Usage} \\
\midrule
B       & \texttt{B label}        & Unconditional branch to \texttt{label} (always jumps) \\
B\texttt{<cond>} & \texttt{B<cond> label}  & Conditional branch based on flags \\
BL      & \texttt{BL label}       & Branch with link: calls a subroutine, storing return address in \texttt{LR}. \\
BX      & \texttt{BX Rm}          & Branch to address in register, often \texttt{BX LR} to return from a subroutine. \\
CBZ     & \texttt{CBZ Rn, label}  & Branch if \texttt{Rn == 0}. Example: \texttt{CBZ R0, Done}. \\
CBNZ    & \texttt{CBNZ Rn, label} & Branch if \texttt{Rn != 0}. Example: loop until counter reaches zero. \\
\bottomrule
\end{tabularx}
\end{table}

\paragraph{\texttt{CBZ/CBNZ}} instructions have specific constraints:
\begin{itemize}[nosep]
  \item \textbf{Register}: operand must be a low register \texttt{R0--R7}.
  \item \textbf{Range}: branch is \emph{forward-only}; the destination must be within 0--126 bytes after the instruction.
  \item \textbf{Flags}: does not update condition flags (\texttt{N, Z, C, V}).
\end{itemize}
For backward or longer jumps, use \texttt{CMP}/\texttt{TST} with conditional branches (\texttt{BEQ}, \texttt{BNE}, \texttt{BGT}, \dots).
\subsubsection{How Branch Instructions Work}

Branch instructions change the flow of execution by modifying the Program Counter (\texttt{PC}). 
When a branch is executed, the instruction encodes an \emph{offset} which is added to the current value of the \texttt{PC}.

\paragraph{Offset calculation:}  
The branch instruction contains a signed immediate value (positive or negative).  
The processor adds this offset (aligned to halfword boundaries) to the current \texttt{PC}.  
\begin{itemize}
    \item A \emph{positive offset} causes a \textbf{forward branch} (jump to a higher memory address, later in the program).  
    \item A \emph{negative offset} causes a \textbf{backward branch} (jump to a lower memory address, earlier in the program).  
\end{itemize} 

\paragraph{Example:}  
Suppose a branch instruction is located at address \texttt{0x100}, and the assembler encodes an immediate offset of \texttt{-0x08}.  
The effective target address will be:
\[
0x100 + 4 + (-0x08) = 0xFC
\]
This means the processor jumps \textbf{backward} to an earlier instruction.  
Such negative offsets are typically used to implement loops (e.g., repeat until zero).


\subsubsection{Condition Codes}
Conditional branches use condition codes that test the processor status flags (N, Z, C, V) set by previous instructions. These conditions enable implementation of high-level constructs like if-statements and loops.
\begin{table}[H]
\centering
\caption{Common ARM Condition Codes}
\small
\begin{tabularx}{0.72\linewidth}{@{}l l X@{}}
\toprule
\textbf{Cond.} & \textbf{Meaning} & \textbf{Description} \\
\midrule
EQ  & Equal                     & Execute if $Z=1$. \\
NE  & Not equal                 & Execute if $Z=0$. \\
CS/HS & Carry set / Unsigned higher or same & Execute if $C=1$. \\
CC/LO & Carry clear / Unsigned lower        & Execute if $C=0$. \\
MI  & Minus (negative)          & Execute if $N=1$. \\
PL  & Plus (non-negative)       & Execute if $N=0$. \\
VS  & Overflow set              & Execute if $V=1$. \\
VC  & Overflow clear            & Execute if $V=0$. \\
HI  & Unsigned higher           & Execute if $C=1$ \emph{and} $Z=0$. \\
LS  & Unsigned lower or same    & Execute if $C=0$ \emph{or} $Z=1$. \\
GE  & Greater or equal (signed) & Execute if $N=V$. \\
LT  & Less than (signed)        & Execute if $N\neq V$. \\
GT  & Greater than (signed)     & Execute if $Z=0$ \emph{and} $N=V$. \\
LE  & Less or equal (signed)    & Execute if $Z=1$ \emph{or} $N\neq V$. \\
AL  & Always                    & Always execute (default if no condition). \\
NV  & Never                     & Reserved / do not use. \\
\bottomrule
\end{tabularx}
\end{table}
\subsection{Loop Implementation}
Loops are fundamental control structures that repeat a block of code based on conditions. ARM assembly implements loops using combinations of compare instructions, conditional branches, and counters.

\subsubsection{For Loop Structure}
A typical for loop has the structure: initialization, condition check, body execution, and increment/decrement. This type of loop executes a known number of times.
\begin{lstlisting}[caption={Declaring Array and Length}]
        AREA M_DATA, DATA, READONLY
array   DCD 10, 20, 30, 40, 50   ; array of 5 integers
length  EQU 5                    ; number of elements (; just a constant, no memory)
\end{lstlisting}
\newpage
\begin{lstlisting}[caption={For loop implementation pattern}]
        ; Initialization
        MOV     R0, #0                  ; i = 0
        MOV     R1, #0                  ; sum = 0
        LDR     R3, =array              ; load base address of array into R3

for_start
        ; Condition check
        CMP     R0, #length             ; compare i with length
        BGE     for_end                 ; if i >= length, exit loop

        ; Loop body
        LDR     R2, [R3, R0, LSL #2]    ; load array[i]; EA: R3 + (R0 * 4)
        ADD     R1, R1, R2              ; sum += array[i]

        ; Increment
        ADD     R0, R0, #1              ; i++
        B       for_start               ; repeat

for_end
\end{lstlisting}
\subsubsection{While Loop Structure}
While loops check the condition before executing the loop body, potentially executing zero times if the initial condition is false. This type of loop is useful when the number of iterations is not known in advance and depends on dynamic conditions.
\begin{lstlisting}[caption={Declaring Null-Terminated String}]
        AREA M_DATA, DATA, READONLY
mystring DCB "Hello World!", 0    ; null-terminated string
\end{lstlisting}
\paragraph{Note:} \texttt{0} and \texttt{'0'} are two different values, as the former is actually zero, while the latter is the ASCII code for the character '0' (which is 48 in decimal).
\begin{lstlisting}[caption={While loop with string processing example}]
    ; Intialization
    LDR     R0, =mystring   ; pointer to string
    MOV     R1, #0          ; character count = 0

while_start
                            ; Condition check
    LDRB    R2, [R0], #1    ; load current character and post-increment pointer
    CMP     R2, #0          ; check for null terminator
    BEQ     while_end       ; if zero, exit loop
    
                            ; Loop body - do something with R2

    
    B       while_start     ; repeat
while_end
\end{lstlisting}

\subsection{Procedures (Subroutines)}

Procedures are reusable blocks of code that encapsulate a specific task. They promote modular design, code reuse, and clearer program structure.  
In ARM assembly, procedures are implemented using branch-and-link instructions (\texttt{BL}, \texttt{BLX}) along with register usage conventions defined by the ARM Architecture Procedure Call Standard (AAPCS).

\subsubsection{Basic Structure}
A procedure is entered with a \texttt{BL} (branch-with-link) instruction, which stores the return address in the link register \texttt{LR}. The callee returns by branching to \texttt{LR} (e.g., \texttt{BX LR}). By the AAPCS, the first four arguments are passed in \texttt{R0--R3} and the primary return value is placed in \texttt{R0}. 

\paragraph{Example:} simple procedure that expects two integers in \texttt{R0} and \texttt{R1} and returns their sum in \texttt{R0}.

\begin{lstlisting}[caption={Basic procedure structure}]
AddTwo  PROC
        ADD R0, R0, R1   ; return R0+R1 in R0
        BX  LR
        ENDP
\end{lstlisting}
\paragraph{Note:} The \texttt{PROC} and \texttt{ENDP} directives are assembler-specific and may vary between assemblers. They help define the start and end of a procedure for readability and organization and could be omitted if not supported.

\subsubsection{ARM Architecture Procedure Call Standard (AAPCS)}

The AAPCS is the set of rules that define how functions exchange data and how registers must be preserved during a procedure call:

\begin{itemize}[nosep]
  \item \textbf{R0--R3}: Hold the first four parameters. \texttt{R0} also holds the return value. Caller-saved.
  \item \textbf{Stack}: Any additional parameters beyond the first four are passed on the stack.
  \item \textbf{R4--R11}: Must be preserved by the callee. If a procedure uses them, it must save and restore them. 
  \item \textbf{SP (R13)}: Stack pointer, always points to the current top of the stack.
  \item \textbf{LR (R14)}: Link register holds the return address. Caller-saved.
\end{itemize}
\paragraph{Note:} Callees are the procedures being called, while callers are the ones calling the procedure.

\subsection{Stack Management}

The stack is a memory region used to hold return addresses, local variables, and saved registers.  
On the Cortex-M4, the stack is implemented as a \emph{full descending stack}: it grows from high memory addresses to low addresses, and the stack pointer (\texttt{SP}) always points to the last stored value.

\paragraph{Stack Terminology}
\begin{itemize}[nosep]
    \item \textbf{Full stack}: The stack pointer (\texttt{SP}) points to the last used location.  
          The memory at the address of \texttt{SP} contains valid data.
    \item \textbf{Empty stack}: The stack pointer (\texttt{SP}) points to the next free location.  
          The memory at the address of \texttt{SP} is unused.
    \item \textbf{Ascending stack}: The stack grows toward higher memory addresses (not used in ARM Cortex-M).
    \item \textbf{Descending stack}: The stack grows toward lower memory addresses (the model used by ARM Cortex-M).
\end{itemize}
\subsubsection{Stack Operations}

The Cortex-M4 provides \texttt{PUSH} and \texttt{POP} instructions that automatically update the stack pointer (\texttt{SP}) and allow saving or restoring multiple registers in one instruction.

\paragraph{PUSH}
\begin{itemize}[nosep]
    \item \textbf{Format:} \texttt{PUSH \{<reglist>\}}
    \item \textbf{Operation:} \texttt{SP} is decremented to create space, then the registers in \texttt{<reglist>} are stored on the stack.
    \item \textbf{Storage order:} Registers are always stored in \emph{ascending register number order}, regardless of how they appear in the reglist.  
          After the operation, the lowest numbered register is at the lowest memory address of the block, and the highest numbered register at the highest address.  
          Example:
\begin{lstlisting}
        PUSH {R4, R0, R2, LR}
\end{lstlisting}
          In memory (from lowest to highest address): R0, R2, R4, LR.  
          The order in the curly braces does not matter.
    \item \textbf{Restriction:} The program counter (\texttt{PC}) cannot be pushed.
    \item \textbf{Effect on SP:} \texttt{SP} decreases by 4 bytes per register pushed.
\end{itemize}



\paragraph{POP}
\begin{itemize}[nosep]
    \item \textbf{Format:} \texttt{POP \{<reglist>\}}
    \item \textbf{Operation:} Registers in \texttt{<reglist>} are restored from the stack, then \texttt{SP} is incremented.
    \item \textbf{Load order:} Registers are always loaded in ascending register number order, not in the order written.
    \item \textbf{Example:}
\begin{lstlisting}
        POP {R4, R0, R2}
\end{lstlisting}
          Restores R0, then R2, then R4 (in register number order).
    \item \textbf{Special case:} If \texttt{PC} is included, the loaded value becomes the new program counter, effectively returning from the procedure.
    \item \textbf{Effect on SP:} \texttt{SP} increases by 4 bytes per register popped.
\end{itemize}


\subsubsection{Nested Procedure Calls}

When one procedure calls another, the link register (\texttt{LR}) must be preserved, otherwise the return address would be lost. This is done by pushing \texttt{LR} onto the stack before making another call.

\begin{lstlisting}[caption={Nested procedure example}]
OuterProc
        PUSH    {LR}            ; Save return address
        BL      InnerProc       ; Call inner procedure
        MOV     R1, R0          ; Use return value
        POP     {PC}            ; Return to caller

InnerProc 
        MOV     R0, #42         ; Return value
        BX      LR              ; Return
\end{lstlisting}

\newpage
\section{Procedure}

\subsection{Examples}

\subsubsection{Example 1: Array Example — Find Maximum Element}
This example demonstrates how to find the maximum element in an array using a standard for loop structure.
\lstinputlisting[caption={Find maximum element in an array}]{snippets/assembly/exp3/example1.asm}
\paragraph{Check:} Verify that the maximum element is correctly identified and stored in \texttt{MAXRES}.
\newpage

\subsubsection{Example 2: String Example — Count Uppercase Letters}
This example demonstrates how to process a null-terminated string and count the number of uppercase letters (A-Z) using a while-loop structure.
\lstinputlisting[caption={Count uppercase letters in a string}]{snippets/assembly/exp3/example2.asm}
\paragraph{Check:} Verify that the program correctly counts the uppercase letters and stores the result in \texttt{UPPERCOUNT}.

\newpage
\subsubsection{Example 3: Stack Example — Nested Uppercase Counter}
This example demonstrates a nested call: \texttt{CountUpperNested(ptr)} scans a null-terminated string and calls \texttt{IsUpper(ch)} for each character. It shows saving/restoring \texttt{LR} and using a callee-saved register (\texttt{R4}) for the running count.
\lstinputlisting[caption={Nested procedure call to count uppercase letters}]{snippets/assembly/exp3/example3.asm}
\paragraph{Check:} Verify that \texttt{UPPERCOUNT} contains \texttt{5} for the test string.

\subsection{Tasks}

\subsubsection{Task 1: Count Vowels in a String}
Implement procedures to process strings with the following requirements:
\begin{itemize}[nosep]
    \item Create a procedure \texttt{CountVowels} that takes a string pointer in R0 and returns the number of vowels (a, e, i, o, u) in R0.
    \item Use nested procedure calls where \texttt{CountVowels} calls a helper procedure \texttt{IsVowel}.
    \item Follow AAPCS conventions for parameter passing and register usage.
\end{itemize}

\subsubsection{Task 2: Factorial Calculation (Iterative)}
Implement a procedure to calculate the factorial of a non-negative integer:
\begin{itemize}[nosep]
    \item Create a procedure \texttt{Factorial} that takes a non-negative integer in R0 and returns its factorial in R0.
    \item Use an iterative approach with a loop to compute the factorial.
    \item Ensure proper handling of edge cases, such as 0! = 1.
    \item Follow AAPCS conventions for parameter passing and register usage.
\end{itemize}
\subsubsection{Task 3: Factorial Calculation (Recursive)}
Implement a recursive version of the factorial calculation:
\begin{itemize}[nosep]
    \item Create a procedure \texttt{FactorialRec} that takes a non-negative integer in R0 and returns its factorial in R0.
    \item Use recursion to compute the factorial, ensuring proper base case handling.
    \item Manage the stack appropriately to save and restore registers as needed.
    \item Follow AAPCS conventions for parameter passing and register usage.
    \item Test the procedure with various inputs to verify correctness.
\end{itemize}